\documentclass[../report/main.tex]{subfiles}
 
\begin{document}

These sections are the minimum of what we need to cover based on the Project 4 guidelines. Feel free to add additional sections!

% The asterix after \subsection disables section numbering
\subsection*{Algorithm Description}

The 2-opt algorithm is a simple local search algorithm that has application to solve the traveling salesman problem. The main idea behind it is to take a route that crosses over itself and reorder that route so that it no longer overlaps. We encountered this algorithm through it's Wikipedia article \url{https://en.wikipedia.org/wiki/2-opt}.

\subsection*{Algorithm Discussion}

A brief discussion of why you chose this algorithm.

I chose this algorithm because:
\begin{itemize}
	\item It was conceptually easy to implement
	\item It is relatively efficient to compute the path improvement for a given path swap, which means it can be run quickly
	\item Our other algorithms were constructive, while 2-OPT is a local search heuristic (it improves upon an existing path)
	\item Related to above, this means we can combine 2-OPT with an efficient constructive algorithm (nearest neighbor in this case)
\end{itemize}

\subsection*{Algorithm Pseudo-code}

\begin{verbatim}
TSP_TO_ADJMATRIX(Text input file):
    Reads the input file and returns an adjacency matrix from the resulting graph

TSP_COMPUTE_EUCLID_DIST(x1, y1, x2, y2):
    dx = x2 - x1
    dy = y2 - y1
    dist = sqrt(dx * dx + dy * dy)
    return round(dist)

TSP_CREATE_IN_ORDER_TOUR(adj_matrix, tour, num_pts):
    tour[0] = 0
    num_in_tour = 1

    while num_in_tour < num_pts:
        tour[num_in_tour] = num_in_tour
        num_in_tour += 1

    tour_length = TSP_COMPUTE_TOUR_DISTANCE(adj_matrix, tour, num_pts)

    return tour_length

TSP_COMPUTE_TOUR_DISTANCE(adj_matrix, tour, num_pts):
    tour_length = 0

    for i = 0 to num_pts - 1:
        tour_length += adj_matrix[tour[i]][tour[i + 1]]

    tour_length += adj_matrix[tour[num_pts - 1]][tour[0]]

    return tour_length

MAIN():
    
\end{verbatim}

\end{document}